\subsubsection{Architectural patterns/styles}
\begin{flushleft}
\textbf{Three-tier Model:}
The architectural responsibilities that need to be addressed for the BUZZ system to be fully operational can be implemented in a multi-tiered fashion. This can be referred to as the model-view-controller approach or the three-tier approach which consists of persistence, logic/processing/communication and presentation. This approach is ideal for the BUZZ system as it is an on-line forum web application that will make use of each tier for related aspects of the system. This approach allows various users to make use of BUZZ through various clients such as a smartphone and a desktop computer. An example to highlight an advantage of using the three-tier system approach can be a user being able to access BUZZ on various devices. The model and the control layer will remain unchanged while the view will have to be modified to accommodate these clients.  This setup will ultimately help with integration with various systems and it will help with portability between various clients.
\end{flushleft}

\begin{flushleft}
\textbf{REST:}
 Since the BUZZ system will be in the form of an online discussion forum, a constant connection between server and client is not needed. Typically a user will request an action to be performed or simply request a page for viewing. A user may spend time only reading a thread and not actually utilize the network in which case the server should only wait for a request, deliver it, and then wait again. In this way we can us the REST technique to lighten the burden on the server side and free up unnecessary network traffic.
\end{flushleft}
