\documentclass[a4paper,12pt]{article}

\begin{document}

\subsubsection{Scope of Architectural Responsibilities}


\begin{flushleft}
\textbf{Persistence:}
The BUZZ system will need to make use of a database system that will store all the forum related information such as the posts, threads, users and statistical information. For this a relational database management system (RDBMS) will need to be used. Use of a RDBMS has its advantages such as efficient querying thanks to effective indexing of information. The information that is stored in the database may need to be backed-up as well. Persistence will be apart of the back-end of the BUZZ system.
\end{flushleft}

\begin{flushleft}
\textbf{Logic/Communication:}
Use of the Buzz system entails communication between the client and the server. The information that is communicated between these two must not be dependent on what type of client is or server is in use. The communication needs to be standardised. This can be achieved with the use of a protocol across all clients and servers of the BUZZ system. Communication will be apart of the back-end and the front-end of the BUZZ system.
\end{flushleft}

\begin{flushleft}
\textbf{Presentation:}
The BUZZ system must have a user friendly graphic user interface available on all web browsers. Presentation will be apart of the front-end of the BUZZ system.
\end{flushleft}

\subsubsection{Architectural patterns/styles}
The architectural responsibilities that need to be addressed for the BUZZ system to be fully operational can be implemented in a multi-tiered fashion. This can be referred to as the model-view-controller approach or the three-tier approach which consists of persistence, logic/processing/communication and presentation. This approach is ideal for the BUZZ system as it is an on-line forum web application that will make use of each tier for related aspects of the system. This approach allows various users to make use of BUZZ through various clients such as a smartphone and a desktop computer. An example to highlight an advantage of using the three-tier system approach can be a user being able to access BUZZ on various devices. The model and the control layer will remain unchanged while the view will have to be modified to accommodate these clients.  This setup will ultimately help with integration with various systems and it will help with portability between various clients. Since the BUZZ system will be in the form of an online discussion forum, a constant connection between server and client is not needed. A user may spend time only reading a thread and not actually utilize the network in which case the server should only wait for a request, deliver it, and then wait again. In this way we can us the REST technique to lighten the burden on the server side.

\end{document}
