%Constraints - Section 6
%Michelle Swanepoel 13066294


\begin{flushleft}

The following points explain certain architectural constraints specified by the client. 

\begin{itemize}
	\item\textbf{JavaEE}
	\newline
	Java Enterprise Edition is a computing platform that provides an API and runtime environment for developing and running secure network applications which are scalable and reliable. 			This platform incorporates a design based largely on modular components running on an application server. The software running on JavaEE is developed in Java (primarily). JavaEE is 			the reference architecture to be used.
	\newline

	The software of the system should be using JavaEE. This will be used on the server side.

	\item\textbf{JPA}
	\newline
	Java Persistence API will be used to describe the management of relational data in the server of the system.
	\newline

	This will be used on the server side.

	\item\textbf{JPQL}
	\newline
	Java Persistence Query Language is part of the JPA specification. This will be used in the server to make a query for information stored in a relational database.
	\newline

	This will be used on the server side.

	\item\textbf{JSF}
	\newline
	JavaServer Faces is part of Java EE and is used to build component- based user interfaces for web applications. It was specified by the client that this should be used.
	\newline

	This will be used on the server side.

	\item\textbf{HTML}
	\newline
	HyperText Markup Language will be used to specify the structure of the web documents which will be used to present the application to the user.
	\newline

	This will be used on the client side.

	\item\textbf{Ajax}
	\newline
	JavaScript and XML will be used together to send and receive data to and from the server, asynchronously.
	\newline

	This will be used on the client side.
\end{itemize}

Devices over which the system must be deployable were not specified by the client, thus .....

\end{flushleft}






