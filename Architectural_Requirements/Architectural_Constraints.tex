%Constraints - Section 6
%Michelle Swanepoel 13066294


%Constraints
\subsection{Technology Constraints}
The following points explain certain architectural constraints specified by the client. 
\begin{itemize}
	\item{\textbf{JavaEE}}
	\newline
	Java Enterprise Edition is a computing platform that provides an API and runtime environment for developing and running secure network applications which are scalable and reliable. This platform incorporates a design based largely on modular components running on an application server. The software running on 					JavaEE is developed in Java (primarily). JavaEE is the reference architecture to be used.
	\newline

	The software of the system should be using JavaEE. This will be used on the server side.

	\item{\textbf{JPA}}
	\newline
	Java Persistence API will be used to describe the management of relational data in the server of the system.
	\newline

	This will be used on the server side.

	\item{\textbf{JPQL}}
	\newline
	Java Persistence Query Language is part of the JPA specification. This will be used in the server to make a query for information stored in a relational database.
	\newline

	This will be used on the server side.

	\item{\textbf{JSF}}
	\newline
	JavaServer Faces is part of Java EE and is used to build component-based user interfaces for web applications. It was specified by the client that this should be used.
	\newline

	This will be used on the server side.

	\item{\textbf{HTML}}
	\newline
	HyperText Markup Language will be used to specify the structure of the web documents which will be used to present the application to the user.
	\newline

	This will be used on the client side.

	\item{\textbf{Ajax}}
	\newline
	JavaScript and XML will be used together to send and receive data to and from the server, asynchronously.
	\newline

	This will be used on the client side.
\end{itemize}


\subsection{Other technologies}

The following points explain other technologies that will be used but are not constraints. 
\begin{itemize}
	\item{\textbf{CSS3}}
	\newline
	Cascading Style Sheets will be used to describe the look and formatting of the HTML documents on the client side.
	
	\item{\textbf{JSON}}
	\newline
	JavaScript Object Notation could be used to transmit data between the server and the actual web application.
	
	\item{\textbf{JavaScript:}}
		\begin{itemize}
			\item{\textbf{Jquery}}
			\newline
			jQuery is a JavaScript library which will be used to simplify the client-side scripting of HTML.
			\item{\textbf{jQueryUI}}
			\newline
			jQuery UI is a set of user interface interactions, themes, etc. which is built n top of the jQuery JavaScript Library.
			\item{\textbf{Google Code Prettify}}
			\newline
			Google Code Prettify will be used to highlight snippets of source code in HTML pages.
		\end{itemize}
	
	\item{\textbf{Bootstrap for responsive design}}		
	\newline
	The HTML- and CSS-based templates can be used for typography, forms, buttons, navigation, etc.	

	\item{\textbf{ObjectDB}}
	\newline
	This an object database specifically for Java. It does not provide its own API, thus JPA (mentioned earlier in "Technology Constraints") will be used as the API. Object databases are generally faster and more efficient and should be used when high performance is needed and complex data is presented.
\end{itemize}







