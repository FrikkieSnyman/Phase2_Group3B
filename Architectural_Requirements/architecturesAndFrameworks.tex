% COS 301 - Mini Project
% Phase 2 - Group 3-B


\begin{flushleft}
	Some appropriate API's already exist which suitably complements the implementation of the software.
	
	\subsection{Syntax highlighting}
	Seeing as ons of the priority features include posting code snippets in the BUZZthreads, the google-code-prettify (\url{https://code.google.com/p/google-code-prettify/}) is to be used in order to automatically implement syntax highlighting of the aforementioned code snippets when they are posted. The API supports a wide variety of languages, including C and friends, Java, Python, HTML, CSS, JavaScript and many more applicable languages.
	\subsection{Translation for pluggable requirements}
	The application must be pluggable. Using the google-translate-api (\url{https://cloud.google.com/translate/docs}) 
	automatically translates the content that is found on the application. This means that the application is not restricted to certain language speakers, and enhances its pluggability.
	\subsection{Plagiarism check}
	PlagScan API (\url{https://api.plagscan.com/guide}) is an open-source API that handles plagiarism checks by sending the data to be checked as a POST request. The response sent is dependant on the configuration, and can be set to be received as XML data, and some methods can be received as either plain binary data or HTML. In order to use this API, one must first register for a PlagScan Pro organization account, after which an API key must be generated. The API can also be set up to define from which IP ranges requests can be sent using the generated API key.


\end{flushleft}