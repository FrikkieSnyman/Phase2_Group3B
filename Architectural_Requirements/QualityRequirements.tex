% COS 301 - Mini Project
% Phase 2 - Group 3-B

\begin{flushleft}
	\begin{itemize}
	%Sebastian's Section 
	%Started my points.
	%Still need to add some points when I think of them.
		\item \textbf{Scalability:} 
Scalability is of extreme importance.
\newline

No assumptions can be made on the strength and size of the servers that the BUZZ system will be hosted on and hence need to be scalable at any point in time. We also have to take into account that there will be a small margin of growth in users as the years go by since the University allows for this growth (an estimate of 10\%) in their student admission numbers.
\newline

		\item \textbf{Performance Requirements:} 
Performance is a very important requirement as the system must be able to accomodate thousands of users and threads.
\newline

The system will mostly be run on large servers capable of servicing an entire university, however performance must still be considered.
\newline

		\item \textbf{Maintainability:} 
Maintainability is very important to the system as any failure of a key component could result in loss of data or the inability to use the system.
\newline

	%Kyhle's Section 
	%Started my points.
	%Still need to edit my points at a later stage.
		\item \textbf{Reliability and Availability:} 
The reliability and availability of the system is important for all users, i.e. The students, tutors, lecturers, and administrative staff.
\newline

If the system is unavailable, then users would not be able to access the required information on the system. This could have serious implications on all users. It is thus imperative to minimise system downtime, thus maximising system availability.\newline	
		
		\item \textbf{Security:} 
The security of the system is important for all users, i.e. The students, tutors, lecturers, and administrative staff.
\newline

The purpose of security is to protect the information stored in the system, whether it be the systems information or user data, and prevent unauthorised access to and/or modification of the information.
\newline
 
		\item \textbf{Monitorability and Auditability:} 
The system will be monitored by the administrative staff and users that are specifically assigned the role of maintenance. \newline

This will help ensure that users abide by the netiquette and plagiarism policies.  \newline
			
	%Andreas' Section
		\item \textbf{Testability:} 
		
		Testability is of medium importance in the system. The system needs to be tested before it goes online. \newline
		
		What could be tested for:
		\begin{itemize}
		\item User profiles should be successfully created and have the correct information in them.
		\item Users’ privileges correspond to their user rank.
		\item Security breaches.
		\item All services provide the correct output if given certain input.
		\end{itemize}
		
		\item \textbf{Usability:} 
		
		User experience is an important aspect in the system. The users should be able to easily identify all the elements of the system as well as be able to use them with ease. \newline
		
		Participation will increase if the interface is easy to understand and easy to use. \newline
		
		A navigation bar needs to be implemented so that the users can easily navigate to where they want to be. \newline
		
		Readability is also a very important aspect in the system. Using a font that is easily readable can greatly increase users’ experience. Choosing the correct colour scheme is also important. \newline
		
		Users should be able to:
		\begin{itemize}
		\item Know how to use the system without any serious assistance.
		\item Learn how to use new concepts if the system was to be upgraded.
		\item Remember how to use the system after the first use.
		\item Like the system.
		\item Give feedback. 
		\end{itemize}
		
		\item \textbf{Integrability:} 
		
		Integrability is of medium importance. It will be used for security reasons and portability reasons. \newline
		
		The system will need to be easily integrable with other database software. If the current database software fails then the system will need to be able to switch database software in order to continue logging all incoming and outgoing data. \newline
		
		If the web service fails, then the system must be easily portable to another web service (different versions of software). \newline
		
		The abstract factory pattern can be used to easily port the system to another environment. \newline
	\end{itemize}
	
\end{flushleft}

